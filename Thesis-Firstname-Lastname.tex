\documentclass[a4paper, 12pt, headsepline=true]{scrartcl} % headsepline ist für die Linie unter der Kopfzeile verantwortlich

\usepackage{polyglossia}
\setdefaultlanguage[spelling=new]{german}

\usepackage[automark]{scrlayer-scrpage}
\clearpairofpagestyles
\ihead{\headmark}
\ohead{\pagemark}

\pagestyle{scrheadings}
\setkomafont{pageheadfoot}{\small}

\usepackage[onehalfspacing]{setspace}
\setmainfont{Charis SIL} % set the main body font (\textrm), assumes Charis SIL is installed


\usepackage{csquotes}
\usepackage[style=authoryear,sorting=ynt]{biblatex}
\addbibresource{lit.bib}

\usepackage{enumitem}
\renewcommand{\labelitemiii}{$\star$}
\usepackage{tabularx} %bessere Tabellenumgebung mit variablen Spaltenbreiten und definierbarer fester Tabellenbreite z.B. \textwidth

\usepackage[export]{adjustbox}

\usepackage[printonlyused, withpage]{acronym}

\usepackage[
german,
colorlinks=true,
linkcolor=blue, % einfache interne Verknüpfungen
anchorcolor=black,% Ankertext
citecolor=green, % Verweise auf Literaturverzeichniseinträge im Text
urlcolor=cyan % Farbe der URLs
 % Back-Links zu den Kapiteln
]{hyperref}

\usepackage{graphicx}
\usepackage{wrapfig}

% Umbenennung der Caption Beschreibung unter Bildern von "Abbildung" in "Abb." um Platz zu sparen.
\addto\captionsgerman{%
	\renewcommand{\figurename}{Abb.}%
}

% Definiere eine neue Liste bei der die Unterpunkte auch nummeriert sind, wie bei einem Inhaltsverzeichnis.
\newlist{legal}{enumerate}{10}
\setlist[legal]{label*=\arabic*.}


\begin{document}


\begin{titlepage}
	\begin{figure}
		\includegraphics[width=0.5\textwidth,right]{figures/hska-logo}
		\vspace{3cm}
	\end{figure}


	{\scshape\huge\bfseries Praxisbericht\par}

	\vspace{2cm}
	
	
	\renewcommand{\arraystretch}{1.5}
	\begin{tabularx}{\columnwidth}{XX}
		\textbf{Matrikelnummer:} 		& \textbf{100100100}\\
		Name: 							& Logi Logik\\
		Firma:							& Läuft mit Latex GmbH \\
		Zeitraum:						& 01.03.2016 - 31.08.2016 \\
		Datum:							& 04.11.2016 
	\end{tabularx}
\end{titlepage}

\newpage

% Nötig um in der PDF Datei einen Lesezeicheneintrag für das Inhaltsverzeichnis zu bekommen.
\pdfbookmark[1]{Inhaltsverzeichnis}{toc}
\tableofcontents
\newpage

\addcontentsline{toc}{section}{Abkürzungsverzeichnis}
% Angabe in eckigen Klammern sollte das längste Acronym enthalten.
% Das ist notwendig damit sich der Einschub am längsten Acronym orientiert.
\begin{acronym}[KISS]
	\acro{kiss}[KISS]{Keep it simple stupid}
	
\end{acronym}

\clearpage

\addcontentsline{toc}{section}{Abbildungsverzeichnis}
\listoffigures
\clearpage


\section{Einleitung}

Ein Beispiel die Verwendung von Acronymen, \ac{kiss}. Ist auch ganz interessant bei der zweiten Verwendung: \ac{kiss}.

\subsection{Unternehmensbeschreibung}

\subsection{Zielsetzung \& erwartete Ergebnisse}

\subsection{Strategie der Zielerreichung}

\subsection{Gang der Arbeit}

\section{Vorläufige Gliederung}

Ein Beispiel für die vorläufige Gliederung im Exposé.

\begin{legal}
	\item Einleitung \hfill 1
	\begin{legal}
		\item Zielsetzung \dotfill 2
		\item Methodik der Arbeit \dotfill 2
		\item Erwartete Ergebnisse \dotfill 3
		\item Erläuterung zum Aufbau \dotfill 3
	\end{legal}
	
	
	\item Unternehmen \hfill 4
	\begin{legal}
		\item Vorstellung der Läuft Gmbh \dotfill 4
		\item Zweitens \dotfill 4
		\item Drittens \dotfill 5
	\end{legal}
	
	
	\item Anforderungen an ein modernes Reportingsystem \hfill 7
	\begin{legal}
		\item Reporting im Unternehmensumfeld \dotfill 7
		\item Evaluierung der Verfügbarkeit von Daten  \dotfill 8
		\item Visualisierung im Kontext mobiler Systeme \dotfill 9
		\item Abgrenzung der Arbeit \dotfill 10
	\end{legal}
	
	
	\item Technologieauswahl für ein echtzeitfähiges System  \hfill 11
	\begin{legal}
		\item Definition von Echtzeit \dotfill 11
		\item Vorstellung möglicher technologischer Plattformen  \dotfill 12
		\begin{legal}
			\item P1 \dotfill 12
			\item P2 \dotfill 14
			\item P3 \dotfill 16
		\end{legal}
		\item Erarbeitung von Entscheidungskriterien \dotfill 18
		\item Evaluierung der Plattformen anhand der Entscheidungskriterien \dotfill 20
		\item Entscheidung \dotfill 23
		
	\end{legal}
\end{legal}




\section{Hauptteil}
\section{Fazit}








\printbibliography[heading=bibintoc]

\end{document}
